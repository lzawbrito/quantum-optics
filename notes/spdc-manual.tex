\documentclass{article}
\usepackage[smalltitle,usefancyhdr]{/Users/lzawbrito/latex-templates/lzawbrito-template}

\usepackage[
    sorting=none,
    style=ext-numeric
]{biblatex}  % better than bibtex
\renewcommand*{\citesetup}{%
  \sffamily
  \biburlsetup
  \frenchspacing}
\addbibresource{ref.bib}
\DeclareOuterCiteDelims{cite}{\sffamily\color{mygreen}\bibopenbracket}{\sffamily \color{mygreen}\bibclosebracket}



\setdocnames{Lucas Z. Brito}{Entanglement Lab Instructor Manual}[PHYS2010]

\begin{document}
\maketitle


\begin{tocbox}
	\tableofcontents
\end{tocbox}

\section{Introduction}

\section{Set-up}
Placing the collimators in the proper place is the most difficult part of the 
lab set-up. Proper coupling with the emitted cone requires correct placement 
down to the order of millimeters. One way 

With all optics in place, there are two degrees of freedom that must be tuned 
to 


\section{Theoretical Background}
\subsection{Non-linear optics}
Linear optics 
\begin{equation*}
  P(t) = \epsilon_0 \chi E(t)
\end{equation*}

Nonlinear optics 
\begin{equation*}
  P(t) = \epsilon_0 \left[
    \chi^{(1)} E(t) 
    + \chi^{(2)} E(t)^2
    + \chi^{(3)} E(t)^3
    \right]\qquad 
    \chi^{(n)} \equiv n\text{-th order susc.}
\end{equation*}

$ E(t) = E_0 e^{i\omega t} + \text{c.c.} $
Second harmonic generation from a scalar
\begin{equation*}
  P^{(2)} (t) = \epsilon_0 \chi^{(2)}E(t)^2
    = \epsilon_0 \chi^{(2)} E_0 e^{2i\omega t}
      + \epsilon_0 \chi^{(2)} E_0^\ast e^{-2i\omega t}
        + 2\epsilon_0 \chi^{(2)} E_0 E_0^\ast
\end{equation*}
parametric - quantum mechanical state of crystal is same before and after

\nocite{*}
\addcontentsline{toc}{section}{Bibliography}
\printbibliography[title=Bibliography]

\end{document}